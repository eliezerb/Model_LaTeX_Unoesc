% ----------------------------------------------------------
% Introdução
%Ex: \chapter{TÍTULO A SER IMPRESSO NO CORPO DO TEXTO}{Título no cabeçalho}{Título no Sumario}
% ----------------------------------------------------------
\chapter{INTRODUÇÃO} % se usar o \chapter* ele não vai colocar no sumario
%\addcontentsline{toc}{chapter}{Introdução} inclui manualmente no sumario sem numeração
No mundo, a cada instante, surgem novas tecnologias e avancos nas mais variadas ciencias e areas do conhecimento, pode-se afirmar a impossibilidade de se desenvolver em uma area isoladamente, uma vez que elas interligam-se e explicam-se cada vez mais. O trabalho proposto de conclusao de curso retrata esta realidade ja que este trabalho propoe uma solucao para algumas necessidades do Laboratorio de Materiais e Solos do Curso de Engenharia Civil da Unoesc. A integracao sera responsavel por fazer o monitoramento de amostras de concreto compostas por um ou mais corpos de prova, com aquisicao de dados analogicos da Sala de Preparacao das Amostras. Todo o controle de cilindros e materiais para testes sao feitos em planilhas eletronicas e no papel.

Depois de conhecida e escolhida a area de estudo e de pesquisa do Laboratorio de Materiais e Solos da Engenharia Civil, a primeira acao foi realizar uma entrevista com a professora Angela Zanboni Piovesan, responsavel pelo laboratorio, para entender como funciona o processo de testes e detalhar quais as funcoes basicas que o sistema deve atender. Em sequencia foram estudadas as normas regulamentadoras que regem os testes, com clareza, os procedimentos realizados para ruptura das amostras de concreto.
\section{APRESENTAÇÃO}
\lipsum[1-1]

\section{DESCRIÇÃO DO PROBLEMA}
\lipsum[1-1]

\section{JUSTIFICATIVA}
Nam dui ligulaquet magna, vitae ornare odio metus a mi. Morbi ac orci et nisl hendrerit mollis. Suspendisse ut massa suspendisse ut massa a suspendisse ut massa a suspendisse ut massa.Nam dui ligulaquet magna, vitae ornare odio metus a mi. Morbi ac orci et nisl hendrerit mollis. Suspendisse ut massa suspendisse ut massa a suspendisse ut massa a suspendisse ut massa.

\section{OBJETIVOS}

\subsection{Objetivo geral}
Nam dui ligulaquet magna, vitae ornare odio metus a mi. Morbi ac orci et nisl hendrerit mollis. Suspendisse ut massa suspendisse ut massa a suspendisse ut massa a suspendisse ut massa.

\subsection{Objetivos específicos}
Desdobramento do objetivo geral. \textcolor{red}{Escreva no máximo 5 objetivos.}
 \begin{itemize}
	\item Item 1
	\item Item 2
	\item Item 3
\end{itemize}

 \section{METODOLOGIA}
 \lipsum[1-1]